%%%%%%%%%%%%%%%%%%%%%%%%%%%%%%%%%%%%%%%%%%%%%%%%%%%%%%%%%%%%%%%%%%%
%%% Documento LaTeX 																						%%%
%%%%%%%%%%%%%%%%%%%%%%%%%%%%%%%%%%%%%%%%%%%%%%%%%%%%%%%%%%%%%%%%%%%
% T�tulo:	Comandos
% Autor:  Ignacio Moreno Doblas
% Fecha:  2014-02-01
%%%%%%%%%%%%%%%%%%%%%%%%%%%%%%%%%%%%%%%%%%%%%%%%%%%%%%%%%%%%%%%%%%%
% Tabla de materias:
% 1 Informaci�n del Documento %
% 2 Comandos a nivel de texto %
% 3 Comandos a nivel de entorno %
% 4 Comandos a nivel de p�gina y secci�n %
% 5 Otros comandos %
%%%%%%%%%%%%%%%%%%%%%%%%%%%%%%%%%%%%%%%%%%%%%%%%%%%%%%%%%%%%%%%%%%%

% 1 Informaci�n del Documento %
\newcommand{\pfctitlename}{T�tulo del Proyecto fin de Carrera}
\newcommand{\pfcauthorname}{Nombre del autor}
\newcommand{\pfctutorname}{Nombre del tutor}
\newcommand{\pfcanno}{a�o de publicaci�n}

% 2 Comandos a nivel de texto %
\newcommand{\R}{\textsuperscript{\textregistered}}	%S�mbolo registrado%
\newcommand{\C}{\textsuperscript{\copyright}}	%S�mbolo Copyright%
\newcommand{\TM}{\texttrademark} %S�mbolo Trade Mark (marca comercial)%

% 2.1 Comandos abreviatura %
\newcommand{\tit}{\textit} %Fuente cursiva (it�lica)%
\newcommand{\tbf}{\textbf} %Fuente negrita%
\newcommand{\ttw}[1]{\texttt{#1}} %Fuente m�quina de escribir (typewriter)%
%Combinaci�n%
\newcommand{\textittt}[1]{\textit{\texttt{#1}}} %it�lica y typewriter%
\newcommand{\textittw}{\textittt} % Otra forma de escribirlo.
\newcommand{\tittw}{\textittw} %Shortened%
\newcommand{\tbftw}[1]{\tbf{\ttw{#1}}}

%Crea una nueva l�nea y la indenta sin crear interlineado extra.
\newcommand{\nli}{\\ \indent} 

%Para escribir un correo electr�nico%
\newcommand{\mailto}[1]{\href{mailto:#1}{#1}}

% Si vas a hacer un uso b�sico de \index (entradas en el �ndice de s�lo un nivel, sin formatos especiales, etc.), define la orden
\newcommand{\miindex}[1]{#1\index{#1}}

\newcommand{\hs}{\hspace} % Abreviatura espacio horizontal
\newcommand{\vs}{\vspace} % Abreviatura espacio vertical

% Abreviaturas para los conjuntos de n�meros m�s comunes.
\newcommand{\realnumbers}{\mathbb R}
\newcommand{\naturalnumbers}{\mathbb N}
\newcommand{\integernumbers}{\mathbb Z}
\newcommand{\rationalnumbers}{\mathbb Q}
\newcommand{\complexnumbers}{\mathbb R}
\newcommand{\irrationalnumbers}{\mathbb I}

% Doble barra sobre una letra (para expresar las matrices).
\newcommand{\doublebar}[1]{\bar{\bar{#1}}} 
% Ej: \vector(y) = \doublebar(A) \vector(x) (Stma. lineal de ec.)

% 3 Comandos a nivel de entorno %
\newcommand{\benu}{\begin{enu}} % Begin enumerate
\newcommand{\eenu}{\end{enu}} 	% End enumeration

%Comando para escribir c�digo Python
\newcommand{\code}[3]{
  %\hrulefill
  %\subsection*{#1}
  %\subsubsection{#1}
  \lstinputlisting{#2}
  %#1\\
  \begin{table}[h!]
  	\centering
  	\caption{#1}
  	\label{#3}
  \end{table}
  \vspace{2em}
}

% 4 Comandos a nivel de p�gina y secci�n %
%Crea p�gina en blanco
\newcommand{\blankpage}{\clearpage{\pagestyle{empty}\cleardoublepage}}

% Versi�n x del comando section: sin numeraci�n pero s� aparece en la tabla de contenidos.
\newcommand{\sectionx}[1]{
	\section*{#1}
	\addcontentsline{toc}{section}{#1}
}

% Versi�n y del comando section: sin numeraci�n y NO aparece en la tabla de contenidos.
\newcommand{\sectiony}[1]{
	\section*{#1}
}

% Versi�n x del comando chapter: sin numeraci�n pero s�  aparece en la tabla de contenidos.
\newcommand{\chapterx}[1]{
	\chapter*{#1}
	%\addcontentsline{toc}{chapter}{#1} %Caused by minitoc package%
	\addstarredchapter{#1} %For minitoc package%
}

% substituto del comando \chapter: incluye estilo de p�gina.
\newcommand{\chapterbegin}[1]%
	{%
		\pagestyle{fancy}
		\fancyhead[LE,RO]{\thepage}
		\fancyhead[LO]{Cap�tulo \thechapter. #1}
		%\fancyhead[RE]{Parte \thepart \rightmark} %
		\fancyhead[RE]{\nouppercase{\rightmark}} %
				
		\chapter{#1}
	}

% Versi�n x del comando \chapterbegin: sin numeraci�n y aparece en la tabla de contenidos.
\newcommand{\chapterbeginx}[1]%
	{%
		\pagestyle{fancy}
		\fancyhead[RO,LE]{\thepage}
		\fancyhead[RE,LO]{#1}
		%\fancyhead[LO]{Chapter \thechapter}
		%\fancyhead[RE]{Part \thepart} %
		
		\chapterx{#1}
	}

%Fin de cap�tulo
\newcommand{\chapterend}{\pagestyle{empty}\cleardoublepage \thispagestyle{empty}}
%Si fuera un art�culo en lugar un libro, \clearpage en lugar de \cleardoublepage

% 5 Otros comandos %
%\let\Oldpart\part
%\newcommand{\parttitle}{}
%\renewcommand{part}[1]{\Oldpart{#1}\renewcommand{\parttitle}{#1}} %Header customization%

%Cambiar el t�tulo �ndice de cap�tulo a ``Contenido''.
\renewcommand{\mtctitle}{Contenido}

\dominitoc % Para tablas de contenidos por cap�tulo.

\addto{\captionsspanish}{
	\renewcommand{\listtablename}{�ndice de Tablas}
	\renewcommand{\tablename}{Tabla} } % Por ejemplo, modificar el nombre de 'Cuadro' a 'Tabla'.

\addto{\captionsspanish}{
	\renewcommand{\contentsname}{�ndice} }

%Si se desea cambiar el tipo de letra a Arial
% por cualquier raz�n, descomentar las siguientes
% dos l�neas
%\renewcommand{\rmdefault}{phv} % Arial
%\renewcommand{\sfdefault}{phv} % Arial
	
%\addto{\captionsspanish}{
%	\renewcommand{\partname}{Fase} }

%\addto{\captionsspanish}{%
%    \renewcommand{\refname}{\vspace{-4.5ex}}} % Para que no aparezca el texto 'referencias' en la bibliograf�a.

% Modifica el interlineado
%\renewcommand{\baselinestretch}{1.5}